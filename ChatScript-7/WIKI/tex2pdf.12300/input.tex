\documentclass[]{article}
\usepackage{lmodern}
\usepackage{amssymb,amsmath}
\usepackage{ifxetex,ifluatex}
\usepackage{fixltx2e} % provides \textsubscript
\ifnum 0\ifxetex 1\fi\ifluatex 1\fi=0 % if pdftex
  \usepackage[T1]{fontenc}
  \usepackage[utf8]{inputenc}
\else % if luatex or xelatex
  \ifxetex
    \usepackage{mathspec}
  \else
    \usepackage{fontspec}
  \fi
  \defaultfontfeatures{Ligatures=TeX,Scale=MatchLowercase}
\fi
% use upquote if available, for straight quotes in verbatim environments
\IfFileExists{upquote.sty}{\usepackage{upquote}}{}
% use microtype if available
\IfFileExists{microtype.sty}{%
\usepackage{microtype}
\UseMicrotypeSet[protrusion]{basicmath} % disable protrusion for tt fonts
}{}
\usepackage[unicode=true]{hyperref}
\hypersetup{
            pdfborder={0 0 0},
            breaklinks=true}
\urlstyle{same}  % don't use monospace font for urls
\IfFileExists{parskip.sty}{%
\usepackage{parskip}
}{% else
\setlength{\parindent}{0pt}
\setlength{\parskip}{6pt plus 2pt minus 1pt}
}
\setlength{\emergencystretch}{3em}  % prevent overfull lines
\providecommand{\tightlist}{%
  \setlength{\itemsep}{0pt}\setlength{\parskip}{0pt}}
\setcounter{secnumdepth}{0}
% Redefines (sub)paragraphs to behave more like sections
\ifx\paragraph\undefined\else
\let\oldparagraph\paragraph
\renewcommand{\paragraph}[1]{\oldparagraph{#1}\mbox{}}
\fi
\ifx\subparagraph\undefined\else
\let\oldsubparagraph\subparagraph
\renewcommand{\subparagraph}[1]{\oldsubparagraph{#1}\mbox{}}
\fi

\date{}

\begin{document}

\section{ChatScript}\label{chatscript}

Natural Language tool/dialog manager

ChatScript is the next generation chatbot engine that has won the
Loebner's 4 times and is the basis for natural language company for a
variety of tech startups.

ChatScript is a rule-based engine, where rules are created by humans
writers in program scripts through a process called dialog flow
scripting. These use a scripting metalanguage (simply called a
``script'') as their source code. Here what a ChatScript script file
looks like:

\begin{verbatim}
#
# file: food.top
#
topic: ~food []

#! I like spinach
s: ( I like spinach ) Are you a fan of the Popeye cartoons?
    a: ( yes )  I used to watch him as a child. Did you lust after Olive Oyl?
            b: ( no ) Me neither. She was too skinny.
            b: ( yes ) You probably like skinny models.
    a: ( no ) What cartoons do you watch?
            b: ( none ) You lead a deprived life.
            b: ( Mickey Mouse ) The Disney icon.

#! I often eat chicken
u: ( ![ not never rarely ] I * ~ingest * ~meat ) You eat meat.

#! I really love chicken
u: ( !~negativeWords I * ~like * ~meat ) You like meat.

#! do you eat bacon?
?: (do you eat _ [ ham eggs bacon]) I eat '_0

#! do you like eggs or sushi?
?: (do you like _* or _*) I don't like '_0 so I guess that means I prefer '_1.

#! I adore kiwi.
s: ( ~like ~fruit ![~animal _bear] )  Vegan, you too...

#! do you eat steak?
?: (do you eat _~meat) No, I hate _0.

#! I eat fish.
s: ( I eat _*1 >) 
  $food = '_0 
  I eat oysters.
\end{verbatim}

Above example mentioned in article
\href{https://medium.freecodecamp.com/chatscript-for-beginners-chatbots-developers-c58bb591da8\#.2qdxjuyvs}{How
to build your first chatbot using ChatScript}.

\subsection{Basic Features}\label{basic-features}

\begin{itemize}
\tightlist
\item
  Powerful pattern matching aimed at detecting meaning.
\item
  Simple rule layout combined with C-style general scripting.
\item
  Built-in WordNet dictionary for ontology and spell-checking.
\item
  Extensive extensible ontology of nouns, verbs, adjectives, adverbs.
\item
  Data as fact triples enables inferencing and supports JSON
  representation.
\item
  Rules can examine and alter engine and script behavior.
\item
  Planner capabilities allow a bot to act in real/virtual worlds.
\item
  Remembers user interactions across conversations.
\item
  Document mode allows you to scan documents for content.
\item
  Ability to control local machines via popen.
\item
  Ability to read structured JSON data from websites.
\item
  Postgres and Mongo support for big data or large-user-volume chatbots.
\end{itemize}

\subsection{OS Features}\label{os-features}

\begin{itemize}
\tightlist
\item
  Runs on Windows or Linux or Mac or iOS or Android
\item
  Fast server performance supports a thousand simultaneous users.
\item
  Multiple bots can cohabit on the same server.
\end{itemize}

\subsection{Support Features}\label{support-features}

\begin{itemize}
\tightlist
\item
  Mature technology in use by various parties around the world.
\item
  Integrated tools to support maintaining and testing large systems.
\item
  UTF8 support allows scripts written in any language
\item
  User support forum on
  \href{https://www.chatbots.org/ai_zone/viewforum/44/}{Chatbots.org}
  and \href{https://github.com/bwilcox-1234/ChatScript/issues}{here}
\end{itemize}

\section{Getting started}\label{getting-started}

\subsection{Installation}\label{installation}

Take this project and put it into some directory on your machine
(typically we call the directory ChatScript, but you can name it
whatever). That takes care of installation.

\begin{verbatim}
git clone https://github.com/bwilcox-1234/ChatScript
\end{verbatim}

\subsection{How to run locally on a console (for
developement/test)}\label{how-to-run-locally-on-a-console-for-developementtest}

\subsubsection{Windows}\label{windows}

Go to the BINARIES directory and type \texttt{ChatScript}:

\begin{verbatim}
cd BINARIES && ChatScript
\end{verbatim}

\subsubsection{Linux}\label{linux}

\begin{verbatim}
cd BINARIES && ./LinuxChatScript64 local
\end{verbatim}

\subsubsection{MacOS}\label{macos}

\begin{verbatim}
cd BINARIES && ./MacChatScript local
\end{verbatim}

This will cause ChatScript to load and ask you for a username. Enter
whatever you want. You are then talking to the default demo bot
\texttt{Harry}.

\subsection{How to run as a server (for
production)}\label{how-to-run-as-a-server-for-production}

\subsubsection{Windows}\label{windows-1}

Go to the BINARIES directory and type \texttt{ChatScript\ port=1024}

\begin{verbatim}
cd BINARIES && ChatScript port=1024
\end{verbatim}

\subsubsection{Linux}\label{linux-1}

\begin{verbatim}
cd BINARIES && ./LinuxChatScript64
\end{verbatim}

\subsubsection{MacOS}\label{macos-1}

\begin{verbatim}
cd BINARIES && ./MacChatScript
\end{verbatim}

This will cause ChatScript to load as a server. But you also need a
client (to test client-server communication). You can run a separate
command window and go to the BINARIES directory and type
\texttt{ChatScript\ client=localhost:1024} if you are on Windows, or
type \texttt{./LinuxChatScript64\ client=localhost:1024} if you are on
Linux. This will cause ChatScript to load as a client and you can talk
to the server.

\subsection{How to build the engine.}\label{how-to-build-the-engine.}

On windows if you have Visual Studio installed, launch
\texttt{VS2010/chatscript.sln} or \texttt{VS2015/chatscript.sln} and do
a build. The result will go in the \texttt{BINARIES} directory.

On Linux, go stand in the SRC directory and type \texttt{make\ server}
(assuming you have make and g++ installed). This creates
BINARIES/ChatScript, which can run as a server or locally. There are
other make choices for installing PostGres or Mongo.

\subsection{How to build a bot}\label{how-to-build-a-bot}

Run ChatScript locally. From the ChatScript command prompt, type

\begin{verbatim}
:build Harry
\end{verbatim}

or whatever other preinstalled bot exists. If you have revised basic
data, you can first do \texttt{:build\ 0} .

\section{Full Documentation}\label{full-documentation}

\href{/WIKI/README.md}{ChatScript Wiki (user guides, tutorials, papers)}

\section{Contributing}\label{contributing}

\begin{enumerate}
\def\labelenumi{\arabic{enumi}.}
\tightlist
\item
  Fork it
\item
  Create your feature branch (git checkout -b my-new-feature)
\item
  Commit your changes (git commit -am `Add some feature')
\item
  Push to the branch (git push origin my-new-feature)
\item
  Create new Pull Request
\end{enumerate}

\section{Last releases}\label{last-releases}

\href{/changes.md}{changes.md}

\section{Author}\label{author}

\begin{itemize}
\tightlist
\item
  Bruce Wilcox
\item
  home website:
  \href{http://www.brilligunderstanding.com}{BrilligUnderstanding.com}
\item
  mail: \href{mailto:gowilcox@gmail.com}{\nolinkurl{gowilcox@gmail.com}}
\end{itemize}

\end{document}
